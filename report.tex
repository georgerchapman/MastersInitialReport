\documentclass[11pt]{article}
\usepackage{geometry} % to change the page dimensions
\geometry{a4paper} % or letterpaper (US) or a5paper or....
% \geometry{margin=2in} % for example, change the margins to 2 inches all round
% \geometry{landscape} % set up the page for landscape
%   read geometry.pdf for detailed page layout information

% \usepackage[parfill]{parskip} % Activate to begin paragraphs with an empty line rather than an indent

% Extension packages providing additional functionality
\usepackage{amsmath}       % additional math environments
\usepackage{graphicx}      % graphics import from external files
\usepackage{epstopdf}      % automates .eps to .pdf conversion
% epstopdf package may require --shell-escape option to pdflatex
\usepackage{booktabs}      % table typesetting additions
\usepackage{siunitx}       % number and units formatting
\usepackage{caption}       % customisation of captions
\usepackage{url}           % format url addresses
\usepackage{abstract}
\usepackage{psfrag}
\usepackage{fancyhdr}
\usepackage{lastpage}
\usepackage{titling}
% allows formatting of abstract
\usepackage{tikz,pgfplots} % diagrams and data plots
\usepackage{float}
\usepackage{textcomp}
\usepackage[toc,titletoc,page]{appendix}
\usepackage{bm}
\usepackage{comment}

% remove abstract title
\renewcommand{\abstractname}{}

% for referencing
\newcommand{\figref}[2][\figurename~]{#1\ref{#2}}
\newcommand{\tabref}[2][\tablename~]{#1\ref{#2}}
\newcommand{\secref}[2][Section~]{#1\ref{#2}}
\newcommand{\subsecref}[2][Section~]{#1\ref{#2}}
\newcommand{\subsubsecref}[2][Section~]{#1\ref{#2}}

\author{George Chapman}
\title{Investigating the statistics of walkers confined to corrals}

\begin{document}

\maketitle

\begin{abstract}
It has been previously shown that a walking droplet on a vibrating fluid bath demonstrates a macroscopic analogy to a quantum particle.  This analogy has led to a series of experiments where quantum mechanical results have been observed at a macroscopic level.  This analogous macroscopic system therefore may help to further our understanding of the quantum domain, where it is much harder to observe the results of experiments.  The preposed experiment is to observe the statistical dependence of walkers when they are confined to corrals of vaarious shapes, analogous to a confined quantum particle.  The results will be compared to the natural Faraday wave-modes of the corrals and also the predicted results for a confined quantum particle.
\end{abstract}

\section{Introduction}
\label{sec:introduction}

Quantum mechanics is the statistical theory that governs the interaction between systems.  These systems are typically on the scale of nanometres, where the effects of quantum mechanics become most significant.  It is therefore unexpected to observe quantum-like effects and experimental results on a macroscopic scale.

However, a recent line of experiments initially conducted by Couder et al.\cite{1} have exhibited results that are remarkably similar to the well-known results of quantum mechanics.  Using a droplet of silicon oil bouncing on the surface of a silicon oil bath, a droplet can be forced into a horizontal motion.  Using this setup, several well-known quantum-mechanical experiments have been recreated, where the `walking' droplet becomes analogous to a quantum particle \cite{1,6,7}.  These experiments demonstrate that there is a coupling between the droplet and the waves created on the surface of the oil due to its bouncing, where the waves 'guide' the motion of the droplet, producing the quantum-like results.

This system is reminiscent of a theory of quantum mechanics originally proposed by De Broglie and later developed by Bohm \cite{17}, where quantum particles are `guided' by the wave-function of the system and are described as having deterministic trajectories, contrary to the well-known Copenhagen interpretation where the outcome of experiments can only be described statistically.  The system of a walking droplet therefore provides an insightful analogy to the De Broglie-Bohm theory of quantum mechanics.

This report aims to further investigate this analogy between the macroscopic system of a walking droplet and the De Broglie-Bohm theory of quantum mechanics.  An experiment performed by Bush et al.\cite{12} has shown that walking droplets confined to corrals of the order of the Faraday wavelength of the silicon oil exhibit seemingly random motion.  This motion can be viewed statically, with the resulting probability of the walker's position resembling that of a confined quantum particle.  This result will first be replicated under similar conditions, followed by experimentation with corrals of varying shape and size.  The results will be compared to the Faraday modes of the corrals and also the theoretical quantum-mechanical results of particles confined to quantum corrals.  A numerical model of the confined droplet will also be created to study the effect of small variations in the dimensions of the corrals.

\begin{figure*}[t]
    \centering
    \includegraphics[trim={25mm 15mm 25mm 15mm},clip,width=0.8\textwidth]{PhaseDiagram.pdf}
    \caption{Phase diagram showing the various regimes of a bouncing droplet of silicon oil as a function of the droplet diameter $D$ and $\gamma_m$, with viscosity $\mu_L=\SI{50}{\milli\pascal\second}$ and driving frequency $\SI{50}{\hertz}$.  In B there is simple bouncing, in PDB period-doubling, in PDC transition to temporal chaos by a period-doubling cascade, in Int the drop has an intermittent behaviour, W is the region of walkers and F the Faraday instability domain where Faraday waves are induced on the free surface \cite{9}.}
    \label{figphasediagram}
\end{figure*}

\section{Background Theory}
\label{secbackgroundtheory}
% Feedback: Structure of section 2 needs improvement - introduce and make logical links between subsections.  Section 2.1 unclear - possibly need to make more use of figure 1 to highlight links, physics and 'thresholds'.

% Restructure:
%     - Non-coalescence of a droplet
%         - How to get a droplet bouncing and breifly why (air dissapation)
%         - Comment that vibration must be below the faraday threshold
%     - Faraday waves
%         - Description of faraday waves
%         - How to find the faraday wavelength and its relavence to the report
%         - Faraday threshold
%     - From bouncing to walking
%         - How a droplet begins to walk as the accelaration is increased.
%         - Walking threshold
%     - Path memory
%         - How path memory arises and the implications (QM like results)

\subsection{Non-coalescence of a droplet}
\label{secnoncoalescenceofadroplet}

When a droplet of liquid is placed on the surface of the same liquid, coalescence is expected.  There is an instantaneous moment before coalescence where the air layer between the droplet and the fluid interface dissipates \cite{2}.  If the liquid is subject to a vertical oscillation of the form $\gamma(t)=\gamma_m\sin\omega_L t$, where $\gamma_m$ is the maximum acceleration and $\omega_L$ is the angular frequency of oscillation, the droplet lifts off the surface before coalescence, replenishing the air layer.  This periodic dissipating and replenishing of the air layer leads to a stable system of a droplet bouncing on the surface of the liquid.

From \figref{figphasediagram}, it can be seen that there exists an upper threshold for $\gamma_m$.  This is the Faraday threshold, $\gamma_m^F$, above which the surface of the liquid becomes spontaneously wavy.

\subsection{Faraday Waves and Threshold}
\label{secfaradaywavesandthreshold}

It is a well known that when subject to an oscillating vertical acceleration the surface of a liquid can become unstable, forming exotic standing waves, known as Faraday waves \cite{20}.  These waves occur with a frequency $f_F$ of half the driving frequency \cite{15}.

The wavelength of the Faraday waves $\lambda_F$ can be determined from the dispersion relation of a viscous fluid \cite{8},
\begin{equation}
    \label{disp_rel}
    f_F^2=\left[\frac{g}{2\pi\lambda_F}+\frac{\sigma}{\rho}\left(\frac{2\pi}{\lambda_F^3}\right)\right]\tanh\left(\frac{2\pi h_0}{\lambda_F}\right)
\end{equation}
where $\sigma$ is the surface tension, $\rho$ is the density, $h_0$ is the height of the fluid and $g$ is the acceleration due to gravity.  $\lambda_F$ will be useful in determining the appropriate size of the corrals to be created.

As mentioned, Faraday waves occur when $\gamma_m\geq\gamma_m^F$.  $\gamma_m^F$ may be found by \cite{9}
\begin{equation}
    \label{faradaythreshold}
    \gamma_m^F=2^\frac{4}{3}\left(\frac{\rho}{\sigma}\right)^\frac{1}{3}\mu_L\omega_L^\frac{5}{3}.
\end{equation}
A droplet would continue to bounce in the presence of Faraday waves, but with the additon of a random horizontal motion.  This would prove unproductive for the subject of this report, therefore $\gamma_m$ must be kept below $\gamma_m^F$.

\subsection{Bouncing to Walking}
\label{secbouncingtowalking}

Using \figref{figphasediagram} once again, the W region is noticed just before $\gamma_m^F$.  This region denotes where the droplet gains a horizontal motion, becoming a so-called `walker' \cite{5}.  This horizontal motion occurs due to the localised Faraday wave that is emitted from each bounce of the droplet.  As the droplet bounces off the liquid surface, it does so off a slightly inclined surface due to the spherical Faraday wave from the previous bounce \cite{9}.  This gives the droplet a horizontal motion which, in a non-obstructed region of a large liquid bath, tends to a linear horizontal motion described by the 1-dimensional equation of motion
\begin{equation}
    \label{eqofmotion}
    m\frac{d^2x}{dt^2}=F^b\sin\left(2\pi\frac{dx/dt}{V_F^\phi}\right)-f^v\frac{dx}{dt}
\end{equation}
where $F^b$ is the force exerted on the drop by the liquid surface, $V_F^\phi$ is the velocity of the Faraday waves and $f^v$ is the effective damping from the dissipation of the air layer during each bounce \cite{9}.

By seeking steady regimes of the system in the limit of small velocities, it can be shown that the bifurcation threshold $F^b_c$ where the droplet gains horizontal motion is
\begin{equation}
    \label{walkingthreshold}
    F^b_c=f^v\left(\frac{V_F^\phi}{2\pi}\right).
\end{equation}
Above this threshold, the velocity of the walker $V_W$ takes a stable solution of
\begin{equation}
\label{walkervelocity}
V_W=\pm\frac{V_F^\phi\sqrt{6}}{2\pi}\sqrt{\left(F^b-F_c^b\right)/F^b}.
\end{equation}

\subsection{Path Memory}
\label{secpathmemory}

It has been discussed that the walking of the droplet arises due to the wave produced from the previous bounce.  In fact, the surface the droplet bounces off, which provides the horizontal motion, is dependent on a superposition of all previous bounces, assuming an appropriate decay of each wave.  This suggests that the droplet-wave association contains a path memory, where the history of the droplets affects its future motion \cite{8}.  This may be modelled by assuming each bounce produces a spherical wave centred on the position of the bounce.  It can then shown that the height of the surface $\xi(\bm{r},t_i)$ at position $\bm{r}$ and time $t_i$ is
\begin{equation}
    \label{pathmemory}
    \xi(\bm{r},t_i)= \sum_{n=-\infty}^{i-1}\Re\left[\frac{A}{\left|\bm{r}-\bm{r_n}\right|^{1/2}}\exp\left(-\frac{t_i-t_n}{\tau}\right)\exp\left(-\frac{\left|\bm{r}-\bm{r_n}\right|}{\delta}\right)\exp\left(i\frac{2\pi\left|\bm{r}-\bm{r_n}\right|}{\lambda_F}+\phi\right)\right],
\end{equation}
where $\Re$ denotes the real part, $A$ is the wave amplitude at each impact, $\tau$ is the typical decay time and $\bm{r_n}$ and $t_n$ are the position and time of previous bounces, respectively.  As $\gamma_m$ is increased towards $\gamma_m^F$, $\tau$ also increases and hence so does the length of the path memory.

In the case of a large liquid container, the horizontal motion of the walker due to this path memory is simply linear, with a well defined velocity as shown in \eqref{walkervelocity}.  But as the droplet comes close to structures is the liquid container (e.q. the side, or a narrow gap), the waves emitted from the bouncing are reflected off the structures, affecting the horizontal motion of the droplet.  This process essentially provides the droplet with information about its surroundings, guiding its trajectory accordingly.

\section{Quantum Mechanical Significance}
\label{sec:quantummechanicalsignificance}
It was first noticed by Couder et al.\cite{1} that the system of a walking droplet is reminiscent of one of the early theories of quantum mechanics, the De Broglie-Bohm Theory \cite{17}.  The theory suggests that a particle in a quantum mechanical system is not only acted on by a `classical' potential $V(\bm{r})$, but also by a `quantum-mechanical' potential $U(\bm{r})$ where
\begin{equation}
    \label{qmpotential}
    U(\bm{r})=\frac{-\hbar^2}{2m}\frac{\nabla^2\left|\psi(\bm{r})\right|}{\left|\psi(\bm{r})\right|}
\end{equation}
where $\hbar$ is the reduced Planck's constant, $m$ is the mass of the particle and $\psi(\bm{r})$ is the wave-function of the system.  The evolution of the system can then be described by the equation of motion
\begin{equation}
    \label{qmeqofmotion}
    m\frac{d^2\bm{r}}{dt^2}=-\nabla\left[V(\bm{r})-\frac{\hbar^2}{2m}\frac{\nabla^2\left|\psi(\bm{r})\right|}{\left|\psi(\bm{r})\right|}\right].
\end{equation}

The key concept of the theory is that the evolution of a system is in fact deterministic, with the wave-function `guiding' the particle.  It is here where parallels are draw between this interpretation of quantum mechanics and the system of a walking droplet.  Both systems involve a particle interacting with a wave, a wave that is dependent on the position, previous positions and the surroundings of the particle.

\section{Macroscopic Quantum Mechanical Observations}
\label{sec:macroscopicquantummechanicalobservations}
As a result of this analogy between a walking droplet and a quantum mechanical particle, experiments have been carried out which subject the walking droplet to macroscopic representations of historically quantum experiments.

\subsection{Single and Double Slits}
\label{sec:singleandcoubleslits}
It has been shown that when a walker is observed passing through single or double slits, the motion within and after the slit appears to be random \cite{1}.  However, if the statistics of many results are analysed, a familiar distribution for the final trajectory of the walker appears, similar to that of a plane-wave passing through the slit.  The resulting amplitude diffraction pattern for a single slit is
\begin{equation}
    \label{singleslitamplitude}
    f(\alpha)=A\left|\frac{\sin[\pi L\sin(\alpha/\lambda_F)]}{\pi L\sin(\alpha/\lambda_F)}\right|
\end{equation}
where $\alpha$ is the angle subtended by the final trajectory and the path straight through the slit and $L$ is the width of the slit.  For the double slit setup with slit separation $d$, the amplitude diffraction pattern is
\begin{equation}
    \label{doubleslitamplitude}
    f(\alpha)=A\left|\frac{\sin\left[\pi L\sin(\alpha/\lambda_F)\right]}{\pi L\sin(\alpha/\lambda_F)}\cos\left[\pi d\sin(\alpha/\lambda_F)\right]\right|.
\end{equation}
These distributions are similar to a quantum-mechanical particle passing through single or double slits ($\lambda_F$ is replaced by the De Broglie wavelength), demonstrating the similarities between a walker being guided by its Faraday wave and a quantum particle being guided by its wave-function.

\subsection{Orbital Quantisation and Splitting}
\label{sec:orbitalquantisationandsplitting}
Following this quantum-like behaviour of walkers in single and double slit experiments, it was also shown that two walkers can enter into stable orbits, with discrete orbital diameters $d_n$ given by
\begin{equation}
    \label{orbit}
    d_n=(n-\epsilon_0)\lambda_F
\end{equation}
where $\epsilon_0$ is a constant and $n=1,2,3,\ldots$ for in-phase bouncing of the droplets and $n=1/2,3/2,5/2,\ldots$ for anti-phase bouncing \cite{5,9,6}.

\begin{figure}[h]
    \centering
    \includegraphics[width=0.65\columnwidth]{LevelSplitting.pdf}
    \caption{The splitting $\delta$ of the orbitals as a function of the angular velocity $\Omega$.  Here, $V_n$ is the velocity of the walker in the $n^{th}$ orbit and $\lambda_v=\lambda_F$.  The solid data points are corotating walkers and the hollow points counterrotating walkers.}
    \label{fig:levelsplitting}
\end{figure}

This quantum-like orbital quantisation led to the further investigation of this system when a force is applied; in this case the Coriolis force due to rotating the bath \cite{6}.  As the rotation of the bath is increased, a linear splitting of the orbitals is seen, with corotating\footnote{walkers orbiting the same direction to the rotation of the bath.} walkers increasing their orbital diameter and counterrotating\footnote{walkers orbiting in the opposite direction to the rotation of the bath.} walkers decreasing their diameter, as shown in \figref{fig:levelsplitting}.  It is shown that there is a relationship between the vorticity $\bm{\Omega}$ and the fluid velocity $\bm{U}$,
\begin{equation}
    \label{vorticity}
    2\bm{\Omega}=\bm{\nabla}\times\bm{U},
\end{equation}
which corresponds to $\bm{B}=\bm{\nabla}\times\bm{A}$ from electromagnetism, where $\bm{B}$ is the magnetic field and $\bm{A}$ is the magnetic vector potential.  Hence, this system is demonstrated to be a macroscopic analogy of Zeeman splitting of atomic levels when subject to an external magnetic field.

% \subsection{Barrier Tunnelling}
% \label{sec:barriertunnelling}
% \cite{7}
% Walkers have also been shown to escape confinement and pass through barriers in a quantum-like, probabilistic manner \cite{7}.  The barriers were created by altering the depth of the silicon oil bath which, as shown by \eqref{walkervelocity}, directly effects $V_W$, as $V_W=V_W(h)$.  It is seen that the probability of escape of a walker depends on $V_W$ and the barrier thickness, creating a system analogous to an alpha particle exhibiting quantum tunnelling to escape from a nucleus.
% key results, dependence of probability on V and barrier height
% no equations needed

\subsection{Particle statistics in a Corral}
\label{sec:particlestatisticsinacorral}
% \begin{figure}[h]
% 	\includegraphics[width=\columnwidth]{WalkerCorral.pdf}
% 	\caption{Probability distribution of a walking droplets position when confined to a circular corral \cite{12}.}
% 	\label{fig:walkercorral}
% \end{figure}
%
% \begin{figure}[h]
% 	\includegraphics[width=\columnwidth]{ElectronCorral.pdf}
% 	\caption{Spatial image of the eigenstates of a quantum corral \cite{21}.}
% 	\label{fig:electroncorral}
% \end{figure}

\begin{figure}[h]
    \centering
    \begin{minipage}{0.45\columnwidth}
        \includegraphics[width=\linewidth]{WalkerCorral.pdf}
        \caption{Probability distribution of a walking droplets position when confined to a circular corral \cite{12}.}
        \label{fig:walkercorral}
    \end{minipage}
    \hspace{1cm}
    \begin {minipage}{0.45\columnwidth}
        \includegraphics[width=\linewidth]{ElectronCorral.pdf}
        \caption{Spatial image of the eigenstates of a quantum corral \cite{21}.}
        \label{fig:electroncorral}
   \end{minipage}
\end{figure}

An experiment carried out by J. Bush et al. \cite{12} has shown that in the long path memory limit, a walkers motion when confined by a corral becomes complex and seemingly random.  A build-up of the walkers position with time however shows that it may depend on the Faraday mode of the corral; the solution to the wave equation with the correct boundary conditions (\figref{fig:walkercorral}).  This statistical dependence of the walker can be compared to the density of states of a two-dimensional electron gas confined to a quantum corral \cite{21}, where measured eigenstates of the corral agree well with those predicted by solving the Schr{\"o}dinger equation for such a corral, as seen in \figref{fig:electroncorral}.

\section{Proposed Experiment}
\label{sec:proposedexperiment}

The proposed experiment is to further the work of J. Bush et al. \cite{12} in investigating the statistics of walkers confined to corrals with dimensions comparable to $\lambda_F$.  In the aforementioned report, only a circular geometry for the corral was investigated.  In this experiment, several corral geometries will be used, investigating in each case whether the statistics of the position of the walker are related to the Faraday mode of the corral.

% corrals to be used
The corrals which will be investigated will have dimensions of a few multiples of $\lambda_F$.  This is to ensure that the effects of the confinement of the the walker will be consequential in the motion of the walker.  For large corrals, the natural damping of the localised Faraday waves would cause edges of the corrals to only become significant as the walker approaches them, as opposed to the small corral case where the entire geometry of the corral affects the motion of the walker continuously.
The corrals will also vary in geometry, with circular, square and irregular geometries all being investigated.  From existing results \cite{12}, it is expected that the motion and statistics of the droplet will be related to the Faraday modes of the corrals, therefore relatively simple geometries will be used initially, allowing the calculation of the Faraday modes for comparison to the walker statistics.  The corrals will be created using a plastic 3-D printer, allowing easy, inexpensive design and redesign of the geometry.  If it becomes apparent that a high degree of precision is required in the dimensions of the corrals, then other options are available (e.g. CNC machining of the corrals).

% liquid (water?, silicon oil)
The liquid that will primarily be used will be silicon oil, as this has been used extensively in previous experiments.  Various viscosities will be experimented with in the range $\SI{5e-3}{\pascal\second}$ to $\SI{0.1}{\pascal\second}$ \cite{9}.  This will allow experimental verification of \eqref{faradaythreshold} and any other dependencies such as droplet size and optimal driving frequency.
Water will also be tried due to its accessibility.  It is expected that obtaining a stable bouncing droplet when using water will be difficult to obtain, although the process will provide insight into the liquid requirements for such a system.

% Initial experiments of bouncing droplets
Initially, a simple setup will be used to induce a bouncing droplet.  A vibration generator will be fed by the amplified signal from a signal generator.  The vibration generator will be vertically accelerating a basin which will contain the liquid.  This basin will be created using 3-D printing and will have a simple square or circular geometry with a flat base, set perpendicular to the acceleration.  Silicon oil and water will then be used to setup a bouncing, then walking droplet by slowly increasing the acceleration of the basin.  The Faraday threshold will also be found for each liquid.  Once this simple setup has shown that walkers may be created, the basin will be replaced with the corrals under investigation in this experiment.

% long path memory requirements
The walker's motion within the corrals will depend on the wave field created by the walker.  This wave field is sustained by vibrating the corrals with an acceleration near the Faraday threshold.  The motion  of the walkers will therefore be investigated with increasing acceleration, where the path memory is also slowly increased.  Results in the long path memory limit are of the most interest as this provides the walker with an entire picture of its surroundings, and these results may be compared to quantum analogies of the system.

% walker tracking
For any statistical analysis of the walkers position to occur, the horizontal motion of the walker must be tracked continuously during its time in the corral.  Ideally, this will be achieved using a digital camera and appropriate motion tracking software, with the camera mounted directly above the corrals so as not to distort the image and avoid any capturing of the vertical bouncing motion of the walker.  Assuming a stable walker-corral setup can be obtained with the walker avoiding coalescence indefinitely, the camera could be left to track the motion of the walker.  The resulting video footage could then be analysed at a later time with the software to determine the walkers position and velocity with time.

% statistical processing
Using the position and velocity of the walker with time, a statistical analysis of the walkers position will be performed with MATLAB in an attempt the produce results similar to that in \figref{fig:walkercorral}.  These results from each corral will be compared to the Faraday modes of the corrals.  The quantum analog of each corral will also be compared to the results, as the walkers confined to corrals are reminiscent of a 2-D particles in a box.  This will provide insight to the extent of the walker's ability to replicate quantum results.

% numerical modelling of walker
Finally, a numerical model for the walkers confined to corrals is desired.  This will primarily be based on \eqref{pathmemory}, where the reflections from the sides of the corral can be modelled as additional wave sources.  This model would enable the easy investigation of small variation in the corral dimensions and liquid density and surface tension.

% The proposed experiment is to further the work of J. Bush et al. in investigating the statistics of walkers confined to corrals with dimensions comparable to $\lambda_F$.  This will be done by inducing walkers in corrals filled with silicon oil, as used in previous experiments, which are subject to a vertical oscillation.  A variety of corral geometries will be investigated, including the original circular corral, ensuring that previous results can be realised \cite{12}.
%
% The predicted outcome is that the statistics of walkers within simple corrals will remain related to the Faraday modes of the corrals.  It is expected that this relationship will also depend heavily on the exact shape of the corrals.
%
% A numerical model of the walker-corral system will also be created, allowing the testing of small variations in the corral geometries.
%
% A comparison of this experimental and simulated data with known quantum systems will provide further testing of the analogy between a walker and a quantum-mechanical particle.

\bibliographystyle{ieeetr}
\bibliography{references}

\end{document}
